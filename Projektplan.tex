\documentclass[a4paper,11pt]{article}

\usepackage[swedish]{babel}
\usepackage[T1]{fontenc}
\usepackage[utf8]{inputenc}
\usepackage{droid}
\usepackage{amsmath}
\usepackage{amsfonts}
\usepackage{graphicx}
\usepackage{fancyhdr}
\usepackage{listings}
\usepackage{float}
\usepackage{pdfpages}
\usepackage{ulem}

\lstset{breaklines=true,showstringspaces=false,frame=none,numbers=left,language=Java}
\lstset{literate=%
{å}{{\r{a}}}1
{ä}{{\"a}}1
{ö}{{\"o}}1
{Å}{{\r{A}}}1
{Ä}{{\"A}}1
{Ö}{{\"O}}1
}

\title{\textbf{Projektplan: Flongout}}
\author{\textbf{Robin Gunning, Mikael Forsberg, Jonathan Yao Håkansson}\\ \\ Kurs DD1339, Introduktion till datalogi, VT 2015\\ Kungliga Tekniska Högskolan\\ Grupp 7\\ rgunning@kth.se, miforsb@kth.se, jyh@csc.kth.se}

\hyphenation{break-out}
\begin{document}



\pagenumbering{gobble}
\maketitle

\begin{abstract}
\noindent
Flongout är ett spel för två spelare. Spelet bygger på element från
flipperspel samt de klassiska spelen ''Breakout'' och ''Pong''.
\end{abstract}
\newpage

\tableofcontents
\newpage
\pagenumbering{arabic}

\section{Programbeskrivning A}
Flongout är ett spel för två personer. Spelet är en blandning av Pong och Breakout (Arkanoid) med flipperpaddlar.
Med hjälp av ''paddlarna'' ska man försöka ''göra mål'', alltså få in bollen bakom motståndaren. Till sin hjälp har man ett antal ''power-ups''
som vänder spelet till sin fördel.

\bigskip
\noindent 
När programmet startas visas en meny där användaren får välja mellan att starta spelet, konfigurera kontroller och avsluta spelet.
Väljer spelaren att konfigurera kontroller visas en skärm där man kan för vardera spelare välja mellan olika sätt att kontrollera paddeln.

\bigskip
\noindent
När spelet startar så visas vardera spelares paddlar vid skärmens kortsidor, den ena till vänster och den andra till höger.
Till en början befinner sig bollen i mitten av spelplanen och när en startknapp trycks ned så inleds en nedräkning. Efter nedräkningen skjuts bollen iväg med slumpvald riktning och hastighet, och spelet är i full gång!
Spelets gång yttrar sig genom att spelarna försöker undvika att släppa in bollen på sin kortsida. Detta undviks genom att rotera och flytta sin paddel för att slå till bollen i önskad riktning.

Enligt ett visst intervall så tillförs ett antal brickor till spelplanen.
Brickorna utgör ett fast hinder för bollen som vid kontakt tvingas till studs. Efter ett visst antal
kontakttillfällen mellan boll och enskild bricka så förstörs brickan, och lämnar därmed spelplanen. Vissa
typer av brickor kan efterlämna en så kallad ''power-up'' då de förstörs. Denna visas grafiskt på skärmen i
den position brickan tidigare fanns, för att sedan förflytta sig i riktning mot den kortsida där den
spelare befinner sig som anslog bollen i det skede precis innan brickan förstördes. Spelaren får
därmed uppgiften att med sin paddel infånga sagd power-up. Om spelaren lyckas fånga in en power-up
så tillförs spelaren någon form av effekt. Det kan till exempel röra sig om att få
en förstorad paddel, att få sina tillslag på bollen förstärkta eller att bollens storlek ändras.

\bigskip
\noindent
Kommer bollen i kontakt med skärmens långsidor studsar bollen tillbaka in på spelplanen. Vid kontakt med
en kortsida lämnar bollen spelplanen, och en poäng tilldelas den spelare som befinner sig vid
motstående kortsida. Efter en kort paus startar en ny nedräkning innan bollen släpps iväg igen. Då en av spelarna har uppnått en viss mängd poäng avslutas matchen genom att
spelarna inte längre tillåts kontrollera paddlarna, bollen plockas bort och en enkel meny visas.
Denna meny tillåter spelarna att välja om de vill starta om eller återgå till huvudmenyn.

\bigskip
\noindent
Det går även att pausa spelet under en pågående omgång. Då spelet pausas visas en meny
snarlik som den som visas då en omgång är slut, utökad med ett menyval för att återuppta spelet.
Om spelet återupptas sker först en nedräkning innan bollen börjar röra sig.

\bigskip
\noindent
Spelarna kan välja mellan en rad olika sätt för att kontrollera paddlarna. Dessa inkluderar
tangentbord, mus och handkontroll.

\begin{figure}[H]
\centering
% \includegraphics[scale=0.5]{test.jpg}
\caption{En mockup av spelet}
\end{figure}

\section{Programbeskrivning B}
Flongout är ett spel för \textit{en till två} personer. Spelet är en blandning av Pong och Breakout
(Arkanoid) med flipperpaddlar. Med hjälp av ''paddlarna'' ska man försöka ''göra mål'', alltså få
in bollen bakom motståndaren. Till sin hjälp har man ett antal ''power-ups'' som vänder spelet till
sin fördel. \textit{Det förekommer även något vi kallar ''power-downs'' som påverkar den egna paddeln
på något negativt vis om man inte lyckas undvika dem.}

\bigskip
\noindent 
När programmet startas visas en meny där användaren får välja mellan att starta spelet, konfigurera
kontroller och avsluta spelet. Väljer spelaren att konfigurera kontroller visas en skärm där man kan
för vardera spelare välja mellan olika sätt att kontrollera paddeln, \textit{samt för vissa av
inmatningsmetoderna konfigurera vilka funktioner som tilldelas olika knappar och joystickaxlar}.

\bigskip
\noindent
När spelet startar så visas vardera spelares paddlar vid skärmens kortsidor, den ena till vänster
och den andra till höger. Till en början befinner sig bollen i mitten av spelplanen och \sout{när en
startknapp trycks ned så inleds en nedräkning} \textit{en nedräkning inleds automatiskt}. Efter
nedräkningen skjuts bollen iväg med slumpvald riktning och hastighet, och spelet är i full gång!
Spelets gång yttrar sig genom att spelarna försöker undvika att släppa in bollen på sin kortsida.
Detta undviks genom att rotera och flytta sin paddel för att slå till bollen i önskad riktning.

Enligt ett visst intervall så tillförs ett antal brickor till spelplanen. Brickorna utgör ett fast
hinder för bollen som vid kontakt tvingas till studs. Efter ett visst antal kontakttillfällen mellan
boll och enskild bricka så förstörs brickan, och lämnar därmed spelplanen. \sout{Vissa typer av
brickor kan efterlämna} \textit{Då en bricka förstörs kan det skapas} en så kallad ''power-up''. Denna visas
grafiskt på skärmen i den position brickan tidigare fanns, för att sedan förflytta sig i riktning
mot den kortsida där den spelare befinner sig som anslog bollen i det skede precis innan brickan
förstördes. Spelaren får därmed uppgiften att med sin paddel infånga sagd power-up. Om spelaren
lyckas fånga in en power-up så tillförs spelaren någon form av effekt. Det kan till exempel röra sig
om att få en förstorad paddel, att få sina tillslag på bollen förstärkta eller att bollens storlek
ändras. \textit{Vissa power-ups (power-downs) har missgynnande effekter och blir därför istället
en företeelse som spelarna försöker undvika.}

\bigskip
\noindent
Kommer bollen i kontakt med skärmens långsidor studsar bollen tillbaka in på spelplanen. Vid kontakt
med en kortsida lämnar bollen spelplanen, och en poäng tilldelas den spelare som befinner sig vid
motstående kortsida. \sout{Efter en kort paus startar en ny nedräkning} \textit{Omedelbart inleds en
ny nedräkning} efter vilken bollen släpps iväg igen. \sout{Då en av spelarna har uppnått en viss
mängd poäng avslutas matchen genom att spelarna inte längre tillåts kontrollera paddlarna, bollen
plockas bort och en enkel meny visas. Denna meny tillåter spelarna att välja om de vill starta om
eller återgå till huvudmenyn.} \textit{Spelet fortsätter så länge spelarna önskar spela. Om spelarna
önskar avsluta spelet trycker de fram pausmenyn, navigerar till menyvalet för avslut och bekräftar.}

\bigskip
\noindent
Det går även att pausa spelet under en pågående omgång. Då spelet pausas visas en meny \sout{snarlik
som den som visas då en omgång är slut, utökad} med ett menyval för att återuppta spelet. Om spelet
återupptas sker först en nedräkning innan bollen börjar röra sig.

\bigskip
\noindent
Spelarna kan välja mellan en rad olika sätt för att kontrollera paddlarna. Dessa inkluderar
tangentbord, \sout{mus} och handkontroll.

\section{Programbeskrivning C}
Vi hade inte planerat att göra någon ''AI'' / datormotståndare, men en enkel sådan tillkom plötsligt
och fungerade så pass bra (speciellt vid speltestning) att vi behöll den. Därmed är Flongout nu ett
spel man även kan spela för sig själv, även om datormotståndaren inte erbjuder någon större
utmaning. ''Power-downs'' kom som förslag på en utav övningarna och innebar en mycket enkel utökning
av systemet med powerups.

Inte heller konfigurering av tangentbords- och knapplayouter var med i den ursprungliga planeringen.
Behovet av detta blev dock snabbt påtagligt då vi under utvecklingsarbetet använde olika modeller av
USB-handkontroll med vissa skillnader i beteende, exempelvis hade de olika ID-nummer för inbördes
motsvarande knappar och joystickaxlar.

Vi implementerade aldrig idén med att använda en startknapp för att få igång nedräkningen. Själva
nedräkningen implementerades först, utan startknapp, och vi blev så pass nöjda med hur det fungerade
att vi lät bli att införa startknappen.

\bigskip
\noindent
Vi valde att inte införa någon poänggräns eller annan form av automatiskt avslutande av spelet.
Valet gjordes antagligen främst på grund av att det inte fanns någon självklar sådan gräns. Mer om
detta följer i avsnittet sammanfattning i slutet av detta dokument.

På grund av tidsbrist implementerades aldrig kontrollmetoderna mus respektive mus och tangentbord.
Mer om detta följer även det i sammanfattningen senare.

\section{Användarbeskrivning A}
Som målgrupp för spelet ser vi personer med viss vana av dator- och tv-spel. Spelet kommer
inte innehålla några egentliga förklaringar och kommer därmed att kräva en viss nivå av
igenkännande hos användaren. Tanken är att spelet är självförklarande och då mängden innehåll
är begränsat antas en person ur målgruppen ha sett och förstått nästan allt efter ett par
provomgångar. Vid första anblick bör en person direkt kunna koppla utseendet hos bollen och
paddlarna till flipperspel, och mer eller mindre direkt förstå deras syfte.

\section{Användarbeskrivning B}
Som målgrupp för spelet ser vi personer med viss vana av dator- och tv-spel. Spelet kommer inte
innehålla några egentliga förklaringar och kommer därmed att kräva en viss nivå av igenkännande hos
användaren. Tanken är att spelet är självförklarande och då mängden innehåll är begränsat antas en
person ur målgruppen ha sett och förstått nästan allt efter ett par provomgångar. Vid första anblick
bör en person direkt kunna koppla utseendet hos bollen och paddlarna till flipperspel, och mer eller
mindre direkt förstå deras syfte.

\section{Användarbeskrivning C}
Inga förändringar skedde vad gäller målgruppsanpassning. Vid de användartester som gjordes verkade
samtliga testpersoner förstå vad spelet gick ut på utan att få någon inledande förklaring, vilken
tyder på att våran målgruppsanpassning har fungerat bra.

\section{Användarscenarier A}
\subsection{Scenario \#1 A}
Bob och Alice har tänkt spela en omgång Flongout. Bob tar fram sin laptop
och sin USB-handkontroll. De två inser dock att de bara har en handkontroll,
och bestämmer sig för att Alice får spela med tangentbordet. Alice startar
Flongout och navigerar sig från huvudmenyn till spelets kontrollinställningar.
Alice ska vara spelare nummer ett, så hon väljer ''tangentbord'' som kontrollmetod
för spelare ett. För Bob, spelare två, väljer hon istället ''handkontroll \#1'', och
ber Bob att trycka på några knappar för att testa handkontrollen. Då Bob testar
några knappar blinkar en liten lampa vid spelare två på skärmen för att visa
vilken handkontroll som har valts. Alice lämnar sedan kontrollinställningarna
och startar spelet.

\subsection{Scenario \#2 A}
Bob och Alice står nu vid spelets startskärm och vid given input av både tangentbord och handkontroll
rör sig respektive paddel. Alice frågar Bob om han är redo att starta och Bob nickar.

Alice trycker på startknappen och en nedräkning börjar för att visa när bollen ska skjutas iväg, bollen skjuts iväg i slumpvald riktning, Alice har otur den här gången och bollen kommer mot hennes kortsida.
Eftersom det är första gången Alice spelar Flongout så är hon inte beredd på i vilken hastighet paddeln rör sig och missar därför att skydda sin kortsida.
Ställningen är nu Bob 1, Alice 0, nedräkningen börjar igen.
Denna gång färdas bollen mot Bobs kortsida och eftersom Bob använder handkontrollen så har han mycket lättare att styra sin paddel,
det gör han genom att ena analoga styrspaken anger position av paddeln och den andra analoga styrspaken anger paddelns vinkel.

Bob returnerar bollen och den studsar i den nedre långsidan av spelplanen och reflekteras mot Alices kortsida.
Även Alice lyckas denna gång returnera bollen i Bobs riktning och vid kontakt med Bobs paddel dyker ett antal brickor upp på spelplanen.
Bollen färdas i rikning mot en av de nytillkomna brickorn och vid kontakt med brickan så förstörs brickan och försvinner, bollen returneras till Bob.
Bob returnerar bollen ännu en gång och denna gång når bollen Alices kortsida, Alice siktar på en bricka med bokstaven P på och träffar den.
Brickan går sönder och ett P ''flyter'' mot Alices kortsida, Alice fångar in det flytande P:et med sin paddel.
Alice märker ingen skillnad, men när Bob slår till bollen mot Alices del av spelplanen så märker hon genast vad som hänt.

När bollen når Alices halva av spelplanen så saktar bollen och och färdas med halva hastigheten, Alice returnerar nu lätt varje boll som Bob skjuter mot henne.
Bob hinner inte skydda sin kortsida och resultatet är nu 1 - 1, spelet avbryts då det ringer på dörren och Alice måste öppna, så Alice trycker på ESC och hamnar i 
spelmenyn med pågående spel pausat medans hon öppnar dörren.
Spelet kan inte fortsätta vid detta tillfälle då Alice har viktigare saker för sig, så Bob väljer att avsluta spelet via spelmenyn och återgår till huvudmenyn.

\section{Användarscenarier B}
\subsection{Scenario \#1 B}
Bob och Alice har tänkt spela en omgång Flongout. Bob tar fram sin laptop och sin USB-handkontroll.
De två inser dock att de bara har en handkontroll, och bestämmer sig för att Alice får spela med
tangentbordet. Alice startar Flongout och navigerar sig från huvudmenyn till spelets
kontrollinställningar. Alice ska vara spelare nummer ett, så hon väljer ''tangentbord'' som
kontrollmetod för spelare ett. För Bob, spelare två, väljer hon istället ''handkontroll \#1'', och
ber Bob att trycka på några knappar för att testa handkontrollen. \sout{Då Bob testar några knappar
blinkar en liten lampa vid spelare två på skärmen för att visa vilken handkontroll som har valts.}
Alice lämnar sedan kontrollinställningarna och startar spelet.

\subsection{Scenario \#2 B}
Bob och Alice står nu vid spelets startskärm och vid given input av både tangentbord och handkontroll
rör sig respektive paddel. Alice frågar Bob om han är redo att starta och Bob nickar.

\sout{Alice trycker på startknappen och en} En nedräkning börjar för att visa när bollen ska skjutas
iväg, bollen skjuts iväg i slumpvald riktning, Alice har otur den här gången och bollen kommer mot
hennes kortsida. Eftersom det är första gången Alice spelar Flongout så är hon inte beredd på i
vilken hastighet paddeln rör sig och missar därför att skydda sin kortsida. Ställningen är nu Bob 1,
Alice 0, nedräkningen börjar igen. Denna gång färdas bollen mot Bobs kortsida och eftersom Bob
använder handkontrollen så har han mycket lättare att styra sin paddel, det gör han genom att ena
analoga styrspaken anger position av paddeln och den andra analoga styrspaken anger paddelns vinkel.

Bob returnerar bollen och den studsar i den nedre långsidan av spelplanen och reflekteras mot Alices
kortsida. Även Alice lyckas denna gång returnera bollen i Bobs riktning och vid kontakt med Bobs
paddel dyker ett antal brickor upp på spelplanen. Bollen färdas i rikning mot en av de nytillkomna
brickorna och vid kontakt med brickan så förstörs brickan och försvinner, bollen returneras till
Bob. Bob returnerar bollen ännu en gång och denna gång når bollen Alices kortsida, Alice siktar på
en bricka \sout{med bokstaven P på} och träffar den. Brickan går sönder och ett P ''flyter'' mot
Alices kortsida, Alice fångar in det flytande P:et med sin paddel \textit{som genast blir nästan
dubbelt så stor!} \sout{Alice märker ingen skillnad, men när Bob slår till bollen mot Alices del av
spelplanen så märker hon genast vad som hänt.}

\sout{När bollen når Alices halva av spelplanen så saktar bollen in och och färdas med halva
hastigheten.} Alice returnerar nu lätt varje boll som Bob skjuter mot henne. Bob hinner inte skydda
sin kortsida och resultatet är nu 1 - 1, spelet avbryts då det ringer på dörren och Alice måste
öppna, så Alice trycker på ESC och hamnar i  spelmenyn med pågående spel pausat medans hon öppnar
dörren. Spelet kan inte fortsätta vid detta tillfälle då Alice har viktigare saker för sig, så Bob
väljer att avsluta spelet via spelmenyn och återgår till huvudmenyn.

\section{Användarscenarier C}
\textit{Scenariotexterna i B-delen ovan har justerats för att enbart spegla sådant som faktiskt
implementerats i spelet.}

\bigskip
\noindent
Vad gäller indikatorlampor för handkontroller på kontrollinställningsskärmen har vi helt enkelt
glömt bort att detta fanns planerat. Gällande den power-up som skulle ha sänkt hastigheten på bollen
så valdes istället andra varianter som var lättare att implementera. Mer om båda dessa områden
följer i sammanfattningen.

\section{Testplan A}
Förutom unit-testning av programkodens olika delar kommer följande moment testas med riktiga användare:

\begin{itemize}
\item Val av kontrollmetoder (se scenario \#1)
\item Start av spelet (se scenario \#2)
\item Spelande av en omgång för att sedan återgå till huvudmenyn
\item Spelande av flera på varandra följande omgångar för att sedan återgå till huvudmenyn
\item Spelande av en eller flera omgångar för att sedan avsluta spelet
\item Avbryta en omgång och starta en ny
\item Avbryta en omgång och avsluta spelet (se scenario \#2)
\end{itemize}

\section{Testplan B}
Förutom unit-testning av programkodens olika delar kommer följande moment testas med riktiga användare:

\begin{itemize}
\item Val av kontrollmetoder (se scenario \#1)
\item Start av spelet (se scenario \#2)
\item Spelande av en omgång för att sedan återgå till huvudmenyn
\item Spelande av flera på varandra följande omgångar för att sedan återgå till huvudmenyn
\item Spelande av en eller flera omgångar för att sedan avsluta spelet
\item Avbryta en omgång och starta en ny
\item Avbryta en omgång och avsluta spelet (se scenario \#2)
\end{itemize}

\section{Testplan C}
Testplanen följdes som planerat. Vissa delar av programkoden var för svåra att unit-testa och saknar
därmed sådana test. Mer om dessa unit-test i sammanfattningen.

\bigskip
\noindent
\subsection{Feedback från användartester}
Användare \#1:
\begin{quote}
\textit{Hmm. Ja, det var lite småsvårt på tangentbordet. Just med att slå paddeln hela vägen fram
Det hade kanske varit nice om man slog hela vägen fram automatiskt när man tryckte upp/ner.
Jo det var min andra feedback nu. Jag kollade inte på menyalternativen iofs. men visa vilka knappar som gör vad.}
\end{quote}

\bigskip

Användare \#2:
\begin{quote}
\textit{Konstigt när man ställer in gamepaden och man måste röra analogstickorna 2 gånger för att det ska gå in, sen helt plötsligt behöver man inte göra det med knapparna.
Om man väljer keyboard så borde man få se vilka knappar som gör vad.}
\end{quote}

\bigskip

Användare \#3:
\begin{quote}
\textit{Hur länge kör man?
Slutar spelet aldrig?
Konstigt när man väljer kontroll, cpu1 och cpu2. Vad är skillnaden?}
\end{quote}

\bigskip

Användare \#4:
\begin{quote}
\textit{Jag förväntade att man skulle få se vilka knappar som gjorde vad i controller setup men istället så råkade jag ställa om hela keyboardet fel.
Skithäftigt med ai, även fast den gör självmål ganska ofta.
Varför är quick spin up ok? alltså i menyn. Det står ingenstans.}
\end{quote}

\bigskip

Användare \#5:
\begin{quote}
\textit{Kul spel men väldigt svårt. 
Jag fattar ingenting av controllersetup.
Bra att man kunde ta bort bakgrunden. 
Borde finnas flera svårighetsgrader på datorn.}
\end{quote}

\subsection{Följder av användartester}
Eftersom flera av testpersonerna efterfrågade information om vilka
knappar som gör vad lade vi till en extra bild som visas innan
spelet börjar. Bilden visar vilka knappar som styr paddeln enligt
den förinställda tangentbordskonfigurationen, tillsammans med en
text som påpekar att konfigurationen går att ändra i menyn.

Det gjordes även ett försök att förbättra detektionen av axlarna
då man konfigurerar en handkontroll. Idén var att införa en exponentialkurva
för axlarna, som skulle ge högre detektionsvärden då man rört axeln långt
från utgångsläget jämfört med den linjära kurva som användes innan. Tyvärr
fungerade detta inte särskilt bra i praktiken och vi gick således tillbaka
till den linjära kurvan.

\section{Programdesign A}
\subsection{Ramverk}
Programmet kommer att baseras på ramverket Slick2D\footnote{http://slick.ninjacave.com/}.

\subsection{Centrala klasser A}
\subsubsection{\texttt{Main} A}
\noindent Programmets huvudklass \texttt{Main} kommer att ärva från
Slick2D:s \texttt{BasicGame} och kommer att ansvara för att starta och
byta mellan programmets olika ''scener'' och att vidarebefodra Slick2D:s
standardhändelser till dessa.

\subsubsection{\texttt{Scene} A}
\noindent De olika ''scener'' som finns i programmet (meny, inställningar,
själva spelandet) kommer att var för sig skrivas som en egen klass
som implementerar interfacet \texttt{Scene}.

\bigskip
\noindent Exempel på viktiga metoder:
\begin{itemize}
\item \texttt{void Scene::init()}
\item \texttt{void Scene::update(Controls[] input)}
\item \texttt{void Scene::render(Graphics g)}
\end{itemize}

\subsubsection{\texttt{MainMenu implements Scene} A}
\noindent Klass för huvudmenyn.

\bigskip
\noindent Exempel på viktiga metoder: se \texttt{Scene}

\subsubsection{\texttt{ConfigurationScreen implements Scene} A}
\noindent Klass för kontrollinställningsskärmen.

\bigskip
\noindent Exempel på viktiga metoder: se \texttt{Scene}

\subsubsection{\texttt{Game implements Scene} A}
\noindent Klass för själva spelandet.

\bigskip
\noindent Exempel på viktiga metoder: se \texttt{Scene}

\subsubsection{\texttt{InGameMenu implements Scene} A}
\noindent Klass för själva spelandet.

\bigskip
\noindent Exempel på viktiga metoder: se \texttt{Scene}, samt
\begin{itemize}
\item \texttt{void InGameMenu::setBackground(Game game)}
\end{itemize}

\subsubsection{\texttt{Controls} A}
\noindent Klassen \texttt{Controls} kommer att modellera en för spelet
idealisk handkontroll. Separata klasser för olika inmatningsmetoder
kommer sedan att fylla instanser av \texttt{Controls} med lämplig data
som sedan kan läsas på ett generellt sätt av spelets olika delar.

\bigskip
\noindent Exempel på viktiga metoder:
\begin{itemize}
\item \texttt{Controls::setLeftStickAngle(float angle)}
\item \texttt{float Controls::getLeftStickAngle()}
\item \texttt{boolean Controls::isButtonOneDown()}
\end{itemize}

\subsubsection{\texttt{Physics} A}
\noindent Klassen \texttt{Physics} kommer att ansvara för de fysikberäkningar
som behövs. 

\bigskip
\noindent Exempel på viktiga metoder:
\begin{itemize}
\item \texttt{void Physics::resolveCollision(Ball b, Paddle p)}
\item \texttt{void Physics::resolveCollision(Ball b, Brick br)}
\item \texttt{void Physics::resolveWallCollision(Ball b)}
\end{itemize}

\subsubsection{\texttt{Brick} A}
\noindent Brickorna skapas i denna klass. Det ska finnas möjlighet att välja färg, hårdhet och om 
brickan har en powerup eller inte.

\bigskip
\noindent Exempel på viktiga metoder:
\begin{itemize}
\item \texttt{float Brick::getHardness()}
\end{itemize}

\subsubsection{\texttt{Paddle} A}
\noindent Klassen för ''paddlarna'' då både player 1 och player 2 behöver en paddel, samt vissa metoder
för att ändra utseende samt attribut av paddeln.

\bigskip
\noindent Exempel på viktiga metoder:
\begin{itemize}
\item \texttt{float Paddle::getAngularMomentum()}
\end{itemize}

\subsubsection{\texttt{Ball} A}
\noindent Klassen för bollen.

\bigskip
\noindent Exempel på viktiga metoder:
\begin{itemize}
\item \texttt{Vec2 Ball::getVelocityVector()}
\item \texttt{void Ball::accelerate(Vec2 v)}
\end{itemize}

\subsubsection{\texttt{Powerup} A}
\noindent Interface för powerups.

\bigskip
\noindent Exempel på viktiga metoder:
\begin{itemize}
\item \texttt{Powerup::applyTo(Ball b)}
\item \texttt{Powerup::applyTo(Paddle p)}
\item \texttt{Powerup::applyTo(Physics ph)}
\end{itemize}

%%%%%%%%%%%%progdesign B
\section{Programdesign B}
\subsection{Ramverk}
Programmet kommer att baseras på ramverket Slick2D\footnote{http://slick.ninjacave.com/}.

\subsection{Centrala klasser B}
\subsubsection{\texttt{Main} B}
\noindent Programmets huvudklass \texttt{Main} kommer att ärva från
Slick2D:s \texttt{BasicGame} och kommer att ansvara för att starta och
byta mellan programmets olika ''scener'' och att vidarebefodra Slick2D:s
standardhändelser till dessa.

\subsubsection{\texttt{Scene} B}
\noindent De olika ''scener'' som finns i programmet (meny, inställningar,
själva spelandet) kommer att var för sig skrivas som en egen klass
som implementerar interfacet \texttt{Scene}.

\bigskip
\noindent Exempel på viktiga metoder:
\begin{itemize}
\item \texttt{void Scene::init}
\item \texttt{void Scene::update}
\item \texttt{void Scene::render}
\end{itemize}

\subsubsection{\texttt{MainMenuScene implements Scene} B}
\noindent Klass för huvudmenyn.

\bigskip
\noindent Exempel på viktiga metoder: se \texttt{Scene}

\subsubsection{\texttt{ControllerSetupScene implements Scene} B}
\noindent Klass för kontrollinställningsskärmen.

\bigskip
\noindent Exempel på viktiga metoder: se \texttt{Scene}

\subsubsection{\texttt{GameScene implements Scene} B}
\noindent Klass för själva spelandet.

\bigskip
\noindent Exempel på viktiga metoder: se \texttt{Scene}

\subsubsection{\texttt{IngameMenuScene implements Scene} B}
\noindent Klass för själva spelandet.

\bigskip
\noindent Exempel på viktiga metoder: se \texttt{Scene}, samt
\begin{itemize}
\item \sout{\texttt{void InGameMenu::setBackground(Game game)}}
\end{itemize}

\subsubsection{\texttt{Controller} B}
\noindent Klassen \texttt{Controller} kommer att modellera en för spelet idealisk handkontroll.
Separata klasser för olika inmatningsmetoder kommer sedan att fylla instanser av \texttt{Controller}
med lämplig data som sedan kan läsas på ett generellt sätt av spelets olika delar.

\bigskip
\noindent Exempel på viktiga metoder:
\begin{itemize}
\item \sout{\texttt{Controls::setLeftStickAngle(float angle)}}
\item \sout{\texttt{float Controls::getLeftStickAngle()}}
\item \sout{\texttt{boolean Controls::isButtonOneDown()}}

\item \texttt{ctrlObj.leftAnalog.setAngle(double angle)}
\item \texttt{double ctrlObj.leftAnalog.getAngle()}
\item \texttt{boolean ctrlObj.buttonOne.isPressed()}
\end{itemize}

\subsubsection{\texttt{Physics} B}
\noindent Klassen \texttt{Physics} kommer att ansvara för de fysikberäkningar
som behövs. 

\bigskip
\noindent Exempel på viktiga metoder:
\begin{itemize}
\item \sout{\texttt{void Physics::resolveCollision(Ball b, Paddle p)}}
\item \sout{\texttt{void Physics::resolveCollision(Ball b, Brick br)}}
\item \sout{\texttt{void Physics::resolveWallCollision(Ball b)}}
\item \texttt{void Physics::step(double time, double divisor)}
\end{itemize}

\subsubsection{\texttt{Brick} B}
\noindent Brickorna skapas i denna klass. \sout{Det ska finnas möjlighet att välja färg, hårdhet och
om brickan har en powerup eller inte.}

\bigskip
\noindent Exempel på viktiga metoder:
\begin{itemize}
\item \sout{\texttt{float Brick::getHardness()}}
\item \texttt{Brick::setHp(int hp)}
\item \texttt{int Brick::getHp}
\end{itemize}

\subsubsection{\texttt{Paddle} B}
\noindent Klassen för ''paddlarna'' då både player 1 och player 2 behöver en paddel, samt vissa
metoder för att ändra \sout{utseende samt} attribut av paddeln.

\bigskip
\noindent Exempel på viktiga metoder:
\begin{itemize}
\item \sout{\texttt{float Paddle::getAngularMomentum()}}
\item \texttt{float Paddle::getAngularVelocity}
\end{itemize}

\subsubsection{\texttt{Ball} B}
\noindent Klassen för bollen.

\bigskip
\noindent Exempel på viktiga metoder:
\begin{itemize}
\item \texttt{Vec2 Ball::getVelocity\sout{Vector}()}
\item \texttt{void Ball::accelerate(Vec2 v)}
\end{itemize}

\subsubsection{\texttt{Powerup} B}
\noindent \sout{Interface} Abstrakt basklass för powerups.

\bigskip
\noindent Exempel på viktiga metoder:
\begin{itemize}
\item \sout{\texttt{Powerup::applyTo(Ball b)}}
\item \sout{\texttt{Powerup::applyTo(Paddle p)}}
\item \sout{\texttt{Powerup::applyTo(Physics ph)}}
\item \texttt{Powerup::applyStaticEffects(Paddle collector, Paddle other, GameScene game))}
\item \texttt{Powerup::removeStaticEffects(Paddle collector, Paddle other, GameScene game))}
\end{itemize}
%%%%%%%%%%%%progdesign B

\section{Programdesign C}
Från ett översiktligt perspektiv så följdes designplanen riktigt bra. Naturligtvis har det blivit
många små förändringar vad gäller namngivning, parametrar och sammanslagning / uppdelning av
metoder.

\section{Tekniska frågor A}
\begin{itemize}
\item Internt koordinatsystem. Skall origo vara i skärmens centrum?
Skall y-koordinaten växa eller avta i riktning uppåt från origo?

\item Egen fysikmotor eller en existerande? (exempelvis Box2d). 
\noindent Frågan är om det är lättare att bygga en egen motor,
eller att sätta sig in i en färdigbyggd. Oftast hanterar inte en färdig fysikmotor enbart kollisioner som 
är det vi behöver, så i detta fall är det nog lättare att bygga en egen. Problemet med en egen fysikmotor är
att få alla småproblem att fungera, det är lätt hänt att bollar ''buggar'' sig igenom väggar och studsar fel etc.

\item Rita egen vektorgrafik eller använda bilder?
\noindent Ska vi rita grafiken i spelet programmatiskt eller vektorgrafik av typ SVG eller bara rastergrafik, t.ex PNG? 
Att rita grafiken programmatiskt är väldigt tidskrävande men det skalar bättre och kvalitén blir bättre.
Kan slick hantera vektorgrafik av typ SVG? Är vissa grafiska objekt värda att programmera? 

\item Nätverksspel eller inte?
\noindent Nätverksspel hade ju självklart varit roligt, men är tidsramen för detta projekt för kort?
Om det finns tid över kanske nätverksspel kan programmeras. Det förhållandevis stora antalet
fysikberäkningar som behövs i spelet gör nätverksspel särskilt svårt att implementera
på ett bra sätt på kort tid.
\end{itemize}

\section{Tekniska frågor B}
\begin{itemize}
\item Internt koordinatsystem. Skall origo vara i skärmens centrum?
Skall y-koordinaten växa eller avta i riktning uppåt från origo?

\item Egen fysikmotor eller en existerande? (exempelvis Box2d). 
\noindent Frågan är om det är lättare att bygga en egen motor,
eller att sätta sig in i en färdigbyggd. Oftast hanterar inte en färdig fysikmotor enbart kollisioner som 
är det vi behöver, så i detta fall är det nog lättare att bygga en egen. Problemet med en egen fysikmotor är
att få alla småproblem att fungera, det är lätt hänt att bollar ''buggar'' sig igenom väggar och studsar fel etc.

\item Rita egen vektorgrafik eller använda bilder?
\noindent Ska vi rita grafiken i spelet programmatiskt eller vektorgrafik av typ SVG eller bara rastergrafik, t.ex PNG? 
Att rita grafiken programmatiskt är väldigt tidskrävande men det skalar bättre och kvalitén blir bättre.
Kan slick hantera vektorgrafik av typ SVG? Är vissa grafiska objekt värda att programmera? 

\item Nätverksspel eller inte?
\noindent Nätverksspel hade ju självklart varit roligt, men är tidsramen för detta projekt för kort?
Om det finns tid över kanske nätverksspel kan programmeras. Det förhållandevis stora antalet
fysikberäkningar som behövs i spelet gör nätverksspel särskilt svårt att implementera
på ett bra sätt på kort tid.
\end{itemize}

\newpage

\section{Tekniska frågor C}
Resultat / svar på tekniska frågor:
\begin{itemize}
\item Internt koordinatsystem.\newline \textbf{Ja!}
\begin{itemize}
\item Skall origo vara i skärmens centrum?\newline \textbf{Ja!}
\item Skall y-koordinaten växa eller avta i riktning uppåt?\newline \textbf{Växa, så att trigonometriska funktioner fungerar som man är van vid dem.}
\end{itemize}

\item Egen fysikmotor eller en existerande?\newline \textbf{Egen. Att sätta sig in i Box2D visade sig vara icke-trivialt.}

\item Rita egen vektorgrafik eller använda bilder?\newline \textbf{Då vi inte lyckades få Slick2D att ladda SVG fick vi nöja oss med PNG-bilder.}

\item Nätverksspel eller inte?\newline \textbf{Det fanns ingen chans att hinna med detta.}
\end{itemize}

Inga nya tekniska frågor tillkom under utvecklingsarbetet.

\section{Arbetsplan A}
Planen till den 8/5, eller så fort som möjligt, är att först ha en förenklad prototyp där vi skriver en enklare variant av spelet (Pong) samtidigt som vi jobbar 
på klasserna som ska vara med i det riktiga spelet (t.ex lägga paddlarna i paddel-klassen).
Även vissa fysikelement kommer vara samma i prototypen och i det färdiga spelet, t.ex bollens kollision med
långsidorna.

\bigskip
\noindent Till veckan därpå, den 15/5 förväntas fysiken till spelet vara klar och nya element såsom brickor
och powerups läggs till i prototypen efterhand.

\bigskip
\noindent Till den 19/5 är planen att spelet ska vara spelbart och inkludera samtliga planerade
inslag (brickor, powerups) för att kunna inleda användartester. Spelet får sedan finslipas
efter användarnas input.

\bigskip
\noindent Den 22/5 är nya användartester planerade. Planen är att få någon annan grupp i indagruppen att användartesta
vårat spel efter eller innan den muntliga lägesrapporten.

\bigskip
\noindent Ända fram till den 25/5 pågår finputsning av spelet baserade på andra omgångens användartester och spelet 
förväntas vara helt klart tills deadline.

\bigskip
\noindent En uppdelning av klasser kommer att ske mellan projektgruppens medlemmar.

\section{Arbetsplan B}
Planen till den 8/5, eller så fort som möjligt, är att först ha en förenklad prototyp där vi skriver
en enklare variant av spelet (Pong) samtidigt som vi jobbar  på klasserna som ska vara med i det
riktiga spelet (t.ex lägga paddlarna i paddel-klassen). Även vissa fysikelement kommer vara samma i
prototypen och i det färdiga spelet, t.ex bollens kollision med långsidorna.

\bigskip
\noindent Till veckan därpå, den 15/5 förväntas fysiken till spelet vara klar och nya element såsom
brickor och powerups läggs till i prototypen efterhand.

\bigskip
\noindent Till den 19/5 är planen att spelet ska vara spelbart och inkludera samtliga planerade
inslag (brickor, powerups) \sout{för att kunna inleda användartester. Spelet får sedan finslipas
efter användarnas input.}

\bigskip
\noindent Den \sout{22/5} 21/5 är användartester planerade. Planen är att \sout{få någon annan grupp
i indagruppen att användartesta vårat spel efter eller innan den muntliga lägesrapporten.} hugga tag
i folk i korridorerna på KTH och \sout{tvinga} låta dem testa spelet. Spelet får sedan finslipas
efter användarnas input.

\bigskip
\noindent Ända fram till den \sout{25/5} 22/5 pågår finputsning av spelet baserade på \sout{andra
omgångens} användartesterna och spelet förväntas vara helt klart tills deadline.

\bigskip
\noindent \sout{En uppdelning av klasser kommer att ske mellan projektgruppens medlemmar.}

\section{Arbetsplan C}
Arbetsplanen följdes i stort. Användartesterna blev fördröjda, och det blev bara en runda istället
för de två som var planerade. Detta beror till viss del på att vi av någon anledning rörde till det
med olika felaktiga datum (övningen 15/5 skedde i själva verket 13/5, var den 19/5 kom ifrån minns
vi inte och den smått viktiga deadline hade vi av någon anledning skjutit fram tre dagar).

\bigskip
\noindent
Vi gjorde aldrig någon egentlig uppdelning av klasser sinsemellan. Istället blev det inledningvis en
uppdelning i form av att vissa främst arbetade med en förenklad prototyp samtidigt som andra
arbetade med de klasser som skulle ingå i den slutgiltiga versionen men som även behövdes till
prototypen, exempelvis fysiken. Halvvägs genom utvecklingsarbetet upphörde denna indelning,
prototypen övergavs då dess funktion --- att driva på utvecklingen av de klasser som behövdes för att
bygga det riktiga spelet --- var uppfylld, och under resterande tid låg allt fokus på den
slutgiltiga versionen.

\section{Sammanfattning}
\subsection{Oimplementerade funktioner}
\subsubsection{Poäng- eller tidsgräns}
Vi implementerade aldrig någon poäng- eller tidsgräns för att automatiskt avsluta en spelomgång.
Hade vi valt \textit{en} sådan gräns hade det varit oerhört trivialt att lägga till, men det bästa
vore förstås om det erbjöds flera olika valbara gränser, vilket tar klart längre tid att göra
samtidigt som effekten på slutprodukten inte blir särskilt stor.

Vid användartesterna så frågades det dock om när spelet tar slut, och vi kommer antagligen att
vid tillfälle lägga in ett par olika valbara gränser.

\subsubsection{Mus som kontrollmetod}
Mycket tid lades på kontrollmetoder, och vi hade planerat att inkludera mus samt kombinationen mus
och tangentbord som valbar kontrollmetod. Detta hanns dock aldrig med, men med tanke på det fina
system som finns på plats kring valbara kontrollmetoder så vore det dumt att inte lägga till muskontroll
vid tillfälle. Det skulle antagligen fungera riktigt bra att flytta paddeln med musen och klicka
på höger och vänster musknapp för att snärta till paddeln uppåt och nedåt.

\subsubsection{Indikatorlampor för inmatning}
Just idén med indikatorlampor för att testa så man valt rätt index på handkontroll hade vi helt
enkelt glömt bort. En annan idé som dök upp under arbetet men valdes bort på grund av tidsbrist var
att kunna rita upp en representation av den idealiserade handkontrollen och visa hur användarens
inmatningar påverkade denna, tillsammans med textetiketter som visar vad de olika delarna av
den idealiserade kontrollen gör i spelet. Detta är en mycket intressant idé och är något vi kommer
försöka lägga till i efterhand. En sådan visualisering skulle sedan kunna användas för samma
funktion som de tidigare tänkta indikatorlamporna.

\subsubsection{Power-up som sänker bollens hastighet}
Om vi utökar systemet för powerups med en metod \texttt{getDynamicEffects} så blir detta väldigt
enkelt att lägga till, och är således något vi planerar att försöka göra i efterhand.

\subsubsection{Saknade unit-tests}
Vissa delar av programkoden, särskilt de grafiska delarna, är svåra att testa utan att ha tillgång
till så kallade ''mock objects''. Vi gjorde en mycket kort undersökning om stödet för sådana hos
ramverket JUnit och kom fram till att det verkade krävas tredjepartslösningar, och valde därför
att skjuta det på framtiden. Det vore dock mycket lärorikt att titta på hur man brukar göra med
''mock objects'' i JUnit och är någonting vi planerar försöka göra framöver.

\subsection{Vidareutveckling av spelet}
Förutom de inslag som nämnts i avsnittet ovan finns det i dagsläget inga konkreta planer på att
vidareutveckla spelet.

\subsection{Har vi lärt oss något?}
\subsubsection{Mikael Forsberg}
Projektarbetet har varit lärorikt på flera olika sätt. Att skriva en vettigt fungerande fysikmotor
är något av en utmaning även för luttrade programmerare. Virtuella bollar har en tendens att
hamna inuti väggar och ramla genom golv, och det är inte sällan lösningen till en kollision genast
leder till en ny.

Att samarbeta intensivt med Git har varit lärorikt och intressant. Jag
har sedan tidigare en del erfarenhet av Git, men har aldrig samarbetat till den grad att så kallade
''branches'' blir användbara.

Att använda en stor IDE, Eclipse, har även det varit lärorikt. Dock kanske inte främst på det sätt
som kursledaren hade tänkt. Visst är det effektivt att kunna byta namn på klasser och variabler med
ett högerklick och några tangenttryckningar, men priset man får betala verkar vara ganska högt. Eclipse
är nämligen så pass komplext att det behövs många olika konfigurationsfiler och stora mängder cachad
metadata för att hålla reda på alltsammans. Ibland verkar dessa filer bli korrupta. Sådana
tillfällen verkar i regel leda till att Eclipse helt enkelt inte startar
längre, tills dess man raderat samtliga konfigurationsfiler och all metadata. Att man sedan får göra
om alla sina personliga inställningar och på nytt importera sina projektfiler för att börja jobba
igen är inte direkt någon effektivitetshöjande företeelse.

\subsubsection{Robin Gunning}
Uppskatta fin och effektiv kod är nåt jag lärt mig, och det får mig att se på min egen kod på ett annat 
sätt. Jag är inte längre nöjd med att bara slänga ihop nånting som funkar, det ska även vara elegant.
Att projektet skulle kännas mer som ett ''riktigt jobb'' har varit bra, och saker har inte blivit riktigt
som jag tänkt mig då jag enbart skrivit en massa småprogram för övningarna innan.

Projektet har gjort det lättare för mig att sätta mig in i nån annans kod och snabbt kunna utgöra vad 
given kod gör.

Jag trodde att man skulle behöva diskutera varenda liten detalj av programmet med de andra för att komma 
fram till hur man ville ha det, men det behövs inte. Gillar jag inte en del av programmet så får jag programmera 
dit ett val så att man kan stänga av det (bakgrunden). Det verkar som om det för det mesta löser sig genom koden
och man behöver inte diskutera varenda liten detalj.

\subsubsection{Jonathan Håkansson}
Jag lärt mig att man kan bygga upp ett spel på många olika klasser. T.e.x att
 man hade counter som en sorts räknare och power kunde inneha många flera olika sorters klasser. La till en extra powerup som gjorde att bollen gick snabbare. 
\end{document}
